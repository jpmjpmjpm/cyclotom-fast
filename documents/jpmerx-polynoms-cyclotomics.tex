
\documentclass{article}

\usepackage[a4paper, dvips]{geometry}
\usepackage[T1]{fontenc}
\usepackage[utf8]{inputenc}
\usepackage[french]{babel}
\usepackage{amsmath,latexsym}
\usepackage{amssymb}
\usepackage{bbm}
\usepackage{hyperref}
\usepackage{enumerate}
\usepackage{xcolor}
\usepackage{pstricks}
\usepackage{pst-node}
\usepackage[amsmath, standard, hyperref]{ntheorem}
%%\usepackage{french}
\usepackage{theoremref}
\usepackage{listings}
\usepackage{algorithm}
\usepackage{algpseudocode}
\usepackage{soul}
\usepackage{xcolor}
\sethlcolor{yellow}

\newcommand{\R}{\mathbbm{R}}
\newcommand{\Pv}{\mathbbm{P}}
\newcommand{\D}{\mathbbm{D}}
\newcommand{\Hv}{\mathbbm{H}}
\newcommand{\Rs}{\mathcal{R}}
\newcommand{\Fs}{\mathcal{F}}
\newcommand{\N}{\mathbbm{N}}
\newcommand{\F}{\mathbbm{F}}
\newcommand{\G}{\mathcal{G}}
\newcommand{\I}{\mathcal{I}}
\newcommand{\T}{\mathcal{T}}
\newcommand{\Ns}{\mathcal{N}}
\newcommand{\Z}{\mathbbm{Z}}
\newcommand{\Zs}{\mathcal{Z}}
\newcommand{\Q}{\mathbbm{Q}}
\newcommand{\Fpol}{\mathcal{F}_{Pol}}
\newcommand{\Flin}{\mathcal{F}/_{\equiv}}
\newcommand{\Ls}{\mathcal{L}}
\newcommand{\A}{\mathcal{A}}
\newcommand{\Ss}{\mathcal{S}}
\newcommand{\Sf}{\mathfrak{S}}
\newcommand{\B}{\mathcal{B}}
\newcommand{\E}{\mathcal{E}}
\newcommand{\K}{\mathcal{K}}
\newcommand{\C}{\mathbbm{C}}
\newcommand{\U}{\mathcal{U}}
\newcommand{\M}{\mathcal{M}}
\newcommand{\ind}{\textrm{ind}}
\newcommand{\ZFC}{\textbf{\mbox{ZFC}}}
\newcommand{\Gal}{\mathrm{Gal}}
\newcommand{\Vect}{\mathrm{Vect}}
\newcommand{\Hom}{\mathrm{Hom}}
\newcommand{\GF}{\mathrm{GF}}
\newcommand{\pgcd}{\mathrm{pgcd}}
\newcommand{\id}{\mathrm{id}}
\newcommand{\car}{\mbox{car}}
\newcommand{\gr}{\mathrm{gr}}
\newcommand{\card}{\mbox{card}}
\newcommand{\Aut}{\mbox{Aut }}
\newcommand{\Ag}{\mathfrak{A}}
\newcommand{\CD}{\textbf{\mbox{CD}}}
\newcommand{\PfN}{\mathcal{P}_f(\N)}
\newcommand{\PcofE}{\mathcal{P}_{cof}(E)}
\newcommand{\OO}{\mathcal{O}}
\newcommand{\inp}{\in_{\varphi}}
\newcounter{question}
\newcounter{subquestion}[question]
\newcounter{lemme}



\theoremstyle{break}                  % passage à la ligne 
\theorembodyfont{\itshape}     % fonte
\newtheorem{theoreme}{Théorème}
\newtheorem{preuve}{Preuve}
\newtheorem{propriete}{Propriété}
\newtheorem{lemme}{Lemme}
\newtheorem{corollaire}{Corollaire}
\newtheorem{axiome}{Axiome}

\newtheorem{exemple}{Exemple}
\newtheorem{remarque}{Remarque}
\newtheorem{convention}{Convention}
\newtheorem{note}{Note}
\newtheorem{representation}{Représentation graphique}

\parindent=0pt

\begin{document}
\lstset{language=python}

\title{Travail d'étude et de recherche - M1 de Mathématiques\
Calcul des polynômes cyclotomiques}
\author{Jean-Philippe MERX}
\date{2025}

\maketitle

\section*{Utilitaires}
\href{https://tex.stackexchange.com/questions/204411/getting-examples-to-look-like-theorems-lemmas-corollaries}{Théorèmes and Co.}


\href{https://www.youtube.com/watch?v=RLdHp_PB_x0}{Vidéo explicative du calcul des polynômes cyclotomiques à la main}.

\href{https://www.youtube.com/watch?v=gnBbm78jz0Y}{Intéressant pour la fonction de Möbius}.

\href{https://jacquescellier.fr/maths/polynomes_cyclotomiques.pdf}{Propriétés des polynômes cyclotomiques}.

\href{https://en.wikipedia.org/wiki/Reciprocal_polynomial}{Self-reciproqual or palindromic polynomials}.

\section*{Résultats variés}
\href{https://arxiv.org/pdf/0810.5496}{Neighboring ternary cyclotomic coefficients differ by at most one}

\href{https://webusers.imj-prg.fr/~pierre.charollois/Charollois_Pbme_cyclotomiques_agreg2015.pdf}{Problème de préparation à l'agrégation sur les polynômes cyclotomiques}. Où l'on prouve entre autre que les coefficients des polynômes cyclotomiques binaires sont dans $\{-1, 0, 1\}$.

\href{https://arxiv.org/pdf/2111.04034}{Une synthèse concernant les coefficients de polynômes cyclotomiques}Théorème
sion
\href{https://math.dartmouth.edu/~carlp/cyclo.pdf}{On the size of the coefficients of the cyclotomic polynomial}


\section*{Introduction}
Rappeler l'article qui est étudié et ce qui est attendu.

\section*{Mathématiques des polynômes cyclotomiques}
\subsection*{Définitions}
Pour un entier $n \ge 1$, on désigne le polynôme $$P_n(X) = X^n - 1 \in \Q[X]$$ et par $U_n$ les racines de $P_n$, c'est à dire les racines $n^{\text{ème}}$ de l'unité. $U_n$ est un sous-groupe du groupe $\mathbb U$ des complexes de module un. C'est un groupe cyclique fini. On peut le montrer en se souvenant qu'un groupe abélien fini $G$ est somme directe de ses sous-groupes $p$-maximaux $G(p)$, c'est à dire de ses éléments qui sont une puissance de $p$. Ici, on peut montrer que $G(p) = \{x \in U_n \mid x^{m_p} = 1\}$ où $m_p$ est l'ordre de $G(p)$ et donc que $G(p)$ est cyclique. On conclut sachant qu'un groupe abélien fini somme directe de groupes d'ordres premiers est cyclique.\\


On désigne dans la suite par $U_n^* \subseteq U_n$ les générateurs de $U_n$, c'est à dire les éléments dont l'ordre est premier avec $n$. Le $n^{\text{ème}}$ polynôme cyclotomique $\Phi_n \in \Q[X]$ est alors défini par:

$$\Phi_n(x) = \prod_{\zeta \in U^*_n} (X - \zeta) = \prod_{\substack{j=1\\ \pgcd(j,n)=1}}^n (X - e^{\frac{2 \pi i}{n}j})$$

et le $n^{\text{ème}}$ polynôme cyclotomique inverse $\Psi_n \in \Q[X]$ par:

$$\Psi_n(x) = \prod_{\zeta \in U_n \setminus U^*_n} (X - \zeta) = \prod_{\substack{j=1\\ \pgcd(j,n) > 1}}^n (X - e^{\frac{2 \pi i}{n}j}) = \frac{X^n - 1}{\Phi_n(X)}.$$

En particulier: $\Phi_1(X) = X-1$ et $\Psi_1(X) = 1$.

\subsection*{Propriétés utiles au calcul des polynômes cyclotomiques}
On s'attache maintenant à décrire et démontrer des propriétés des polynômes $\Phi_n, \Psi_n$ qui vont permettre de les calculer efficacement. 

\begin{propriete}\thlabel{prop:prodphi}
	Pour $n \ge 1$ on a $P_n(X) = X^n-1 = \prod_{ d \mid n} \Phi_d(X)$.
\end{propriete}
En effet, $U_n$ est la réunion des $U_d^*$ pour $d \mid n$ et les $U_d^*$ sont deux à deux disjoints.

\begin{propriete}
	Les polynômes $\Phi_n$ sont à coefficients entiers et unitaires.
\end{propriete}
Si $P, Q$ sont des polynômes à coefficients entiers, et que $Q$ est unitaire, alors il existe des polynômes $A,R$ appartenant à $\mathbb Z[X]$ tels que $P = AQ + R$. De plus, $A$ est unitaire si $P, Q$ le sont. On prouve ce résultat en appliquant l'algorithme de calcul de la division euclidienne de polynômes dans un corps et en remarquant qu'à chaque étape le terme obtenu est à coefficient entier.\\
L'utilisation de la \thref{prop:prodphi} et une récurrence forte permet de conclure que $\Phi_n$ est un polynôme à coefficients entiers unitaire pour $n \ge 1$.

\begin{propriete}[Irréductibilité des polynômes cyclotomiques]
	Les polynômes $\Phi_n \in \mathbb Z[X]$ sont irréductibles.
\end{propriete}
On démontre cette propriété bien qu'elle ne soit pas utile pour les algorithmes utilisés ici pour leur calcul.\\

Soit $\zeta$ une racine primitive $n$-ième de l'unité, $p$ un nombre premier avec $n$ et $f,g \in \mathbb Q[X]$ les polynômes minimaux unitaires de $\zeta$ et $\zeta^p$. Montrons que $f,g$ sont à coefficients entiers. $\mathbb Z$ étant factoriel, $\mathbb Z[X]$ l'est également et $\Phi_n = f_1^{\alpha_1} \cdots f_r^{\alpha_r}$ où $f_1, \dots ,f_r \in \mathbb Z[X]$. $\Phi_n$ étant unitaire, on peut supposer que les $f_i$ le sont aussi. $\zeta$ est racine de l'un des $f_i$, qui est irréductible sur $\mathbb Z$ et donc sur $\mathbb Q$. Par conséquent, $f$ est l'un des $f_i$, $g$ aussi puisque $\zeta^p$ est racine de $\Phi_n$ et $f,g$ divisent $\Phi_n$ dans $\mathbb Z[X]$.\\

Montrons que $f=g$. Dans le cas contraire, comme $f,g$ sont irréductibles, le produit $f \cdot g$ divise $\Phi_n$ dans $\mathbb Z[X]$. $\zeta^p$ étant racine de $g$, $\zeta$ est racine de $g(X^p)$ et $f(X)$ divise $g(X^p)$ dans $\mathbb Q[X]$, donc aussi dans $\mathbb Z[X]$ puisque $f$ est unitaire. On écrit $g(X^p) = f(X)h(X)$ avec $h(x) \in \mathbb Z[X]$.\\

Projetons cette égalité dans $\mathbb F_p$. $g(X) = a_r X^r + \cdots + a_0$ avec $a_i \in \mathbb Z$. Par Frobenius:

$$\overline{g}(X^p) = (\overline{a_r} X^r + \cdots + \overline{a_0})^p = \overline{g}(X)^p = \overline{f}(X) \overline{h}(X).$$ Si $\varphi \in \mathbb F_p[X]$ est un facteur irréductible de $\overline{f}$, $\varphi$ divise $\overline{g}$ et donc $\varphi^2$ divise $\overline{\Phi}_n$ dans $\mathbb F_p[X]$. Ce qui est absurde, puisque $p$ ne divisant pas $n$ et que $\overline{\Phi}_n^\prime(X) = n X^{n-1} \neq 0$, $\overline{\Phi}_n$ ne peut pas avoir de racine double dans un corps de décomposition.\\

Maintenant, si $\zeta^\prime$ est une racine $n$-ième de l'unité, on a $\zeta^\prime = \zeta^m$ avec $m = p_1^{\alpha_1} \cdots p_s^{\alpha_s}$ où $p_i \nmid n$. Une récurrence permet de montrer que $\zeta^\prime$ et $\zeta$ ont le même polynôme minimal. De sorte que $f$ admet toutes les racines primitives de l'unité comme zéro. Comme $f \mid^ \Phi_n$, et que ces deux polynômes sont unitaires, on obtient que $f = \Phi_n$ et donc que $\Phi_n$ est irréductible.\\



Notons que les polynômes cyclotomiques, ne sont pas toujours irréductibles sur les corps finis. Ainsi, sur $\mathbb F_2$:
$$\Phi_7(X)= (X^2+X^2+1)(X^3+X+1).$$\\


Nous définissons maintenant la fonction de Möbius $\mu : \mathbb N^* \to \{-1, 0, 1\}\}$:
\begin{displaymath}
	\mu(n) = :
	\begin{cases}
		1 & \text{si } n = 1\\
		0 & \text{si } n \text{ a un facteur premier carré}\\
		(-1)^r & \text{où } $r$ \text{ est le nombre de facteurs premiers de } $n$\\
	\end{cases}
\end{displaymath}

\begin{propriete}[Deux propriétés de la fonction de Möbius]
	\begin{enumerate}
		\item $\mu$ est multiplicative: si $m,n $ sont premiers entre eux: $\mu(mn)= \mu(m)\mu(n)$.
		\item Si $n >1$, $\displaystyle \sum_{d \mid n} \mu(d) = 0$.
	\end{enumerate}
\end{propriete}
Si $m$ ou $n$ est égal à $1$, la première propriété est évidente. Si $m,n$ sont premiers entre eux, ils n'ont pas de facteur premier commun. Alors $\mu(mn)=\mu(m)\mu(n)=0$ si $m$ ou $n$ a un facteur premier carré. Dans le cas contraire, le nombre de facteurs premiers de $mn$ est la somme des nombres de facteurs premiers de $m$ et $n$.\\

Concernant la seconde propriété, supposons que $n = \displaystyle \prod_{n = 1}^r p_1^{\alpha_1} \cdots p_r^{\alpha_r}$ où $\{p_1, \dots p_r\}$ sont des premiers distincts et $\alpha_1, \dots, \alpha_r$ des entiers supérieurs ou égaux à $1$. Dans la somme $\displaystyle \sum_{d \mid n} \mu(d)$, seuls les diviseurs $d$ de $n$ sans facteur premier carré ont une contribution non nulle, ce qui conduit à l'égalité:
$$\sum_{d \mid n} \mu(d) = \sum_{i = 0}^{r}\sum_{\substack{S \subseteq \{p_1, \dots, p_r\}\\ \left\vert S \right\vert = i}} (-1)^i = \sum_{i = 0}^{r} \binom{r}{i}(-1)^i =(1-1)^r=0.$$

Nous fournissons à présent une autre définition des polynômes cyclotomiques utilisant la fonction de Möbius.

\begin{propriete}[Définition équivalente des polynômes cyclotomiques]
	\begin{align}		
		\Phi_n(X) = \prod_{d \mid n} (X^d - 1)^{\mu(\frac{n}{d})} = \prod_{d \mid n} (1 - X^d)^{\mu(\frac{n}{d})} = \prod_{d \mid n} (1 - X^\frac{n}{d})^{\mu(d)} \tag{2.4}\label{mobphi}\\
		\Psi_n(X) = \prod_{d \mid n, d<n} (X^d - 1)^{-\mu(\frac{n}{d})} = -\prod_{d \mid n, d<n} (1 - X^d)^{-\mu(\frac{n}{d})}\tag{2.5}\label{mobpsi}
	\end{align}
\end{propriete}

La seconde formule est une conséquence immédiate de la première et de l'égalité $\Phi_n(X) \Psi_n(X) = X^n-1$. Pour montrer la première, notons
$$F_n(X) = \prod_{d \mid n} (X^d - 1)^{\mu(\frac{n}{d})} = \prod_{d \mid n} (X^{\frac{n}{d}} - 1)^{\mu(d)}.$$ Il suffit de prouver que $\displaystyle \prod_{ d \mid n} F_d(X) = X^n-1$, puisque l'on obtient alors par récurrence $F_n(X) = \Phi_n(X)$ partant du constat que $F_1(X) = \Phi_1(X) = 1$. Nous avons:

$$\prod_{ d \mid n} F_d(X) = \prod_{ d \mid n} \prod_{d^\prime \mid d} (X^{d^\prime} - 1)^{\mu(\frac{d}{d^\prime})}$$ et allons montrer que pour $d^\prime$ divisant $n$ fixé

$$\sum_{d \in S_d^\prime} \mu\left(\frac{d}{d^\prime}\right)=\begin{cases}1 &\text{si } d= n\\
	0 &\text{autrement }\end{cases}$$ où $S_{d^\prime} = \{d ; d^\prime \mid d \text{ et } d \mid n\}$.
Dans la somme ci-dessus, on peut changer d'indice de sommation en prenant $e = d/d^\prime$ et effectuer la somme sur l'ensemble $\overline{S}_{d^\prime} = \{e ; e \mid \frac{n}{d^\prime}\}$, c'est à dire l'ensemble des diviseurs de $\frac{n}{d^\prime}$. D'où

$$\sum_{d \in S_d^\prime} \mu\left(\frac{d}{d^\prime}\right) = \sum_{e \mid \frac{n}{d^\prime}} \mu(e)$$ et le résultat compte tenu de la seconde propriété ci-dessus de la fonction de Möbius.

Cette définition permettra le calcul des polynômes cyclotomiques en effectuant des multiplications et des divisions par des polynômes du type $X^d - 1$.

Les propriétés suivantes des polynômes cyclotomiques sont importantes pour les algorithmes de calcul. Elles permettent en effet d'une part de limiter les calculs principaux au cas où $n$ est un produit de nombres premiers distincts, et pour ce dernier cas d'élaborer un algorithme récursif pour leur calcul: l'algorithme \texttt{SPS4} de l'article étudié.


\begin{propriete}
	Soient $p,q$ des entiers premiers tels que $p \nmid n$ et $q \mid n$. Alors:
	\begin{align}
		\Phi_{np}(X) &= \frac{\Phi_n(X^p)}{\Phi_n(X)} \tag{2.8a}\label{ndivphi}\\
		\Phi_{nq}(X) &= \Phi_n(X^q) \tag{2.8b}\label{divphi}\\
		\Psi_{np}(X) &= \Psi_n(X^p)\Phi_n(X) \tag{2.8c}\label{ndivpsi}\\
		\Psi_{nq}(X) &= \Psi_n(X^q)	 \tag{2.8d}\label{divpsi}
	\end{align}
\end{propriete}

Remarquons tout d'abord que si $m$ est un entier quelconque
$$\Phi_{nm}(X) \Psi_{nm}(X) = X^{nm}-1 = (X^p)^m-1=\Phi_{n}(X^m) \Psi_{n}(X^m).$$ Si $\ref{ndivphi}$ est vraie on obtient
$$\Psi_{np}(X) = \frac{X^{np}-1}{\Phi_{np}(X)} =\frac{X^{np}-1}{\Phi_{n}(X^p)}\Phi_n(X)=\Psi_n(X^p)\Phi_n(X)$$ montrant $\ref{ndivpsi}$. De manière similaire, on déduit $\ref{divpsi}$ de $\ref{divphi}$.\\

Pour $n$ entier, nous noterons dorénavant $D_n$ l'ensemble des diviseurs de $n$ sans facteurs carré. On peut réécrire \ref{mobphi}
$$\Phi_{n}(X) = \prod_{d \in S_n} (1 - X^\frac{n}{d})^{\mu(d)}.$$ Si $d$ est un diviseur de $np$ sans facteur premier, ou bien $d$ divise $n$, ou bien $d$ est de la forme $d^\prime p$. D'où l'égalité
$$\Phi_{np}(X) = \prod_{d \in S_{n}} (1 - X^\frac{np}{d})^{\mu(d)}
\prod_{dp \in S_{np}} (1 - X^\frac{np}{dp})^{\mu(dp)}.$$ Le premier facteur est $\Phi_{n}(X^p)$. Observons que dans le second, $dp$ étant sans facteurs carré, $d$ est premier avec $p$ et donc $\mu(dp)= \mu(d)\mu(p)= - \mu(d)$. Ce second facteur est simplement l'inverse de $\Phi_{n}$: nous avons prouvé $\ref{ndivphi}$.\\

Il nous reste à prouver $\ref{divphi}$. Puisque $q$ divise $n$, les diviseurs sans facteurs carré de $nq$ sont ceux de $n$ et
$$\Phi_{nq}(X) = \prod_{d \in S_{n}} (1 - X^\frac{nq}{d})^{\mu(d)} = \Phi_n(X^q),$$ qui est le résultat souhaité.\\

Portant notre attention sur le cas où $2$ divise $n$, nous allons obtenir la propriété suivante:


\begin{propriete}\thlabel{prop:oddprime}
	Pour $n > 1$:
	$$\Phi_{2n}(X) = \begin{cases}
		\Phi_n(X^2) & \text{si } $n$ \text{ est pair}\\
		\Phi_n(-X) & \text{sinon}
	\end{cases}$$
\end{propriete}

Le cas $n$ pair est une application immédiate de $\ref{divphi}$. Si $n$ est impair, les polynômes $\Phi_{2n}(X)$ et $\Phi_{n}(-X)$ sont égaux puisqu'ayant les mêmes racines qui sont simples. Ce résultat nous permet de nous concentrer sur le calcul des polynômes cyclotomiques $\Phi_n$ avec $n$ impair.\\

On déduit d'autre part des formules précédentes $\ref{divphi}$ et $\ref{divpsi}$ pour le cas où $q \mid n$ que


\begin{propriete}[Réduction aux cas des entiers sans facteurs carré]\thlabel{prop:nosquare}
	Si $n = q_1^{e_1} \cdots q_k^{e_k}$ où $q_1, \dots, q_k$ sont des premiers distincts:
	\begin{align*}
		\Phi_{n}(X) &= \Phi_{q_1 \cdots q_k}(X^{q_1^{e_1 - 1} \cdots q_k^{e_k - 1}})\\
		\Psi_{n}(X) &= \Psi_{q_1 \cdots q_k}(X^{q_1^{e_1 - 1} \cdots q_k^{e_k - 1}})
	\end{align*}
\end{propriete}

Cette propriété et la précédente permettent de limiter le calcul des polynômes cyclotomiques au cas où $n$ est un produit de nombres premiers impairs distincts.\\


Pour ce faire, supposons que $n = p_1 \cdots p_k$ soit le produit de $k$ nombres premiers \textbf{impairs} distincts. Pour $1 \le i \le k$, notons $m_i = p_1 \cdots p_{i-1}$ et $e_i = p_{i+1} \cdots p_{k}$. En particulier $m_1=e_k=1$ et on note $e_0 = n$. Pour $1 \le i \le k$, nous avons alors $n = e_i p_i n_i$, ainsi que $e_{i-1}=p_i e_i$ et $m_{i+1}=m_i p_i$. À partir de la formule $\ref{ndivphi}$ et de l'identité $\Phi_{m_k}(X) \Psi_{m_k}(X) = X^{m_k}-1$ on obtient
$$\Phi_{n}(X) = - \frac{\Psi_{m_k}(X^{e_k})}{1- X^{n/p_k}} \Phi_{m_k}(X^{e_{k-1}}).$$
Sachant que $\Phi_1(X^{e_0}) = \Phi_1(X^n) = X^n-1$ on montre alors par récurrence:
$$\Phi_n(X)=\prod_{j=1}^{k} - \Psi_{m_j}(X^{e_j}) \prod_{j=1}^{k} (1-X^{n/p_j})^{-1}(1-X^n)$$
ou encore $$\Phi_n(X)=\prod_{j=2}^{k} - \Psi_{m_j}(X^{e_j}) \prod_{j=1}^{k} (1-X^{n/p_j})^{-1}(1-X^n)$$
puisque $\Psi_{m_1}(X^{e_1}) = \Psi_{1}(X^{e_1}) =1$. L'utilisation de la formule $\ref{ndivpsi}$ et une récurrence similaire permet d'écrire $\Psi_n(X)$ comme produit de polynômes cyclotomiques d'ordres inférieurs; ce que l'on résume dans la propriété suivante:


\begin{propriete}[Formules récursives de calcul des polynômes cyclotomiques]
	\begin{align*}
		\Phi_n(X) &=\prod_{j=2}^{k} - \Psi_{m_j}(X^{e_j}) \prod_{j=1}^{k} (1-X^{n/p_j})^{-1}(1-X^n)\tag{3.17}\label{recurphi}\\
		\Psi_{n}(X) &=\prod_{j=1}^{k} \Phi_{m_j}(X^{e_j})\tag{3.25}\label{recurpsi}
	\end{align*}
\end{propriete}

On note dans la suite du document $\varphi(n)$ l'indicatrice d'Euler de $n$.

\begin{propriete}[Palindromie des polynômes cyclotomiques]
	Pour $n > 1$ impair, le polynôme $\Phi_n(X) = \sum_{i=0}^{\varphi(n)} a_i X^i$ est palindromique tandis que $\Psi_n(X) = \sum_{j=0}^{n-\varphi(n)} b_j X^j$ est anti-palindromique, ce qui signifie que
	$$a_i=a_{\varphi(n)-i} \text{ et } b_j=-b_{n-\varphi(n)-j}.$$
\end{propriete}
Si $\omega$ est une racine primitive $n$-ième de l'unité, $\omega^{-1}$ aussi. $n$ étant supposé impair, $\omega \neq \omega^{-1}$ et le produit des racines de $\Phi_{n}$ est égal à $1 = a_0$ puisque $\varphi(n)$ est pair pour $n > 2$. $X^{\varphi(n)} \Phi_n(\frac{1}{X})$ est un polynôme unitaire, dont les racines sont exactement celles de $\Phi_n(X)$, ce qui prouve que $\Phi_n(X)$ est palindromique.\\

De l'égalité $\Phi_n(X)\Psi_n(X) = X^n-1$ il résulte d'une part que $\Psi_n(X)$ est unitaire et d'autre part que $b_0=-1$. On remarque aussi que si $\zeta$ n'est pas une racine primitive $n$-ième de l'unité, alors $\zeta^{-1}$ non plus. $-X^{n-\varphi(n)} \Psi_n(\frac{1}{X})$ est donc un polynôme unitaire, comme $\Psi_{n}(X)$ et qui possède précisément les racines de $\Psi_{n}(X)$. Ces deux polynômes sont donc égaux, et nous concluons que $\Psi_{n}(X)$ est anti-palindromique.


\begin{propriete}[Produit de polynômes palindromiques / anti-palindromiques]
	Le produit de deux polynômes palindromiques ou anti-palindromiques est palindromique.\\
	Le produit d'un polynôme palindromique par un polynôme anti-palindromique est anti-palindromique.
\end{propriete}
Démonstration claire à partir des définitions. 

\section*{Calcul des polynômes cyclotomiques}
Dans cette section, on décrit la méthode utilisée dans l'algorithme SPS4 de l'article pour calculer de manière rapide les polynômes cyclotomiques, ainsi qu'une implémentation effectuée en Python. On commence par quelques considérations techniques.

\subsection*{Éléments techniques à prendre en compte}

Comme mentionné à la section précédente, la \thref{prop:oddprime} et \thref{prop:nosquare} permettent de restreindre les calculs aux polynômes cyclotomiques $\Phi_n$ pour lesquels $n$ est pair et sans facteur carré; hypothèses utilisées dans l'algorithme SPS4. Le calcul d'un polynôme cyclotomique d'ordre $n$ comportant des facteurs carrés s'effectue en utilisant la \thref{prop:nosquare}. Dans ce cas, on obtient des polynômes creux. Par souci d'économie de mémoire, on utilise alors une représentation sous la forme par exemple de dictionnaires plutôt que de simples tableaux. C'est ce qui est réalisé dans l'implémentation  
SAGE \href{https://github.com/sagemath/sage/blob/develop/src/sage/rings/polynomial/cyclotomic.pyx}{\texttt{cyclotomic.pyx}}.\\

On cherche donc à calculer de manière rapide les polynômes cyclotomiques $\Phi_n$ pour lesquels $n$ "est grand". Mais qu'entend-on par grand? Il y a au moins deux éléments importants à considérer.\\

Tout d'abord l'espace mémoire nécessaire pour le stockage d'un polynôme cyclotomique $\Phi_n$. Si $n = p_1 \cdots p_k$ où $p_1, \dots, p_k$ sont $k$ premiers impairs distincts, le degré de $\Phi_n$ est égal à l'indicatrice d'Euler $\varphi(n) = (p_1-1) \cdots (p_k-1)$. Ainsi pour $n = 3 \cdot 5 \cdot 7 \cdot 11 \cdot 13 \cdot 17 \cdot 19 \cdot 23 \cdot 29$, le degré de $\Phi_n$ est égal à $1,021,870,080$ et à supposer que chacun des coefficients de $\Phi_n$ tienne sur 64 bits, on obtient un polynôme nécessitant 8 Go de mémoire. On approche ici des limites de ce qui calculable en mémoire vive sur un ordinateur personnel.\\

Un autre paramètre majeur pour les implémentations est la magnitude des coefficients de $\Phi_n$. On désigne par $A(n)$ la hauteur de $\Phi_n$, c'est à dire la valeur maximale des valeurs absolues des coefficients de $\Phi_n$. Nous verrons que l'algorithme SPS4 n'utilise que des calculs sur les entiers et que seules des additions (et soustractions) sont mises en œuvre. Encore faut-il pour obtenir des calculs rapides que les entiers sur lesquels on travaille puissent-être représentés par des "entiers machine", ce qui veut généralement dire en 2025 sur 64 bits. Si ce n'est pas le cas, on fait face à un dilemme. Ou bien l'on travaille en précision entière infinie (ce qui est le cas par défaut de Python), mais on perd beaucoup en rapidité. Ou bien on teste dans tous les calculs l'overflow, ce qui est mieux, mais cependant plus lent que d'effectuer directement les additions et soustractions. L'écart de rapidité est cependant ici simplement proportionnel. Une troisième méthode est possible. Elle consiste à avoir calculé par un programme où l'on se prémunit de l'overflow le plus petit $n$, nommé $n_{64}$ pour lequel $A(n)$ reste représentable sur 64 bits, puis dans le \textit{programme définitif}, à mettre en place un calcul différent selon que l'on souhaite calculer $\Phi_n$ avec $n \ge n_{64}$ ou $n < n_{64}$.\\


C'est cette troisième stratégie qui est implémentée dans SAGE. \texttt{cyclotomic.pyx} utilise les résultats suivants de Michael Managan:
\begin{itemize}
	\item Pour $n<10163195$, $A(n)$ est inférieure ou égale à 74989473, soit 26.16 bits, et on peut effectuer les calculs en 32 bits.
	\item Pour $n=10163195$, $A(n)=1376877780831$,  soit 40.32 bits.
	\item Si $n < 169,828,113$, $A(n)$ tient sur 60 bits.
	\item Pour $n = 169828113$, on obtient une hauteur égale à $31484567640915734951$ qui nécessite 65 bits pour être représenté. Dans ce dernier cas, \texttt{cyclotomic.pyx} bascule sur l'utilisation de PARI avec le joli warning: \texttt{print("Warning: using PARI (slow!)")}.\\
\end{itemize}

On le voit, la hauteur $A(n)$ est importante pour la mise en œuvre d'un algorithme de calcul rapide des polynômes cyclotomiques. Pour $n \ge 3 \cdot 7 \cdot 13 \cdot 17 \cdot 23 \cdot 37 \cdot 43 = 169828113$, on n'est plus à même d'utiliser l'arithmétique 64 bits des machines. Il est alors nécessaire de faire appel à des méthodes de calcul en arithmétique infinie.\\

On aborde à la section \nameref{sec:hauteur} la question de la hauteur des polynômes cyclotomiques. Revenons pour le moment à l'algorithme \texttt{SPS4}.

\subsection*{Algorithme \texttt{SPS4}}
L'algorithme \texttt{SPS4} de l'article est un récursif basé sur la formule \ref{recurphi} qui permet de calculer un polynôme cyclotomique à partir de polynômes cyclotomiques inverses d'ordres inférieurs et de la formule \ref{recurpsi} qui donne un polynôme cyclotomique inverse comme produit de polynômes cyclotomiques d'ordres inférieurs. L'algorithme est récursif et alterne l'utilisation des polynômes cyclotomiques et cyclotomiques inverses.\\

Prenons par exemple $n = 3 \cdot 5 \cdot 7 = 105$. On obtient successivement les équations:
\begin{align*}
	\Phi_{105}(X) &= \Psi_{15}(X)\Psi_{3}(X^7)(1-X^{105})(1-X^{15})^{-1}(1-X^{21})^{-1}(1-X^{35})^{-1}\\
	\Psi_{15}(X) &= \Phi_{5}(X) \Phi_{1}(X^3)\\
	\Psi_{3}(X^7) &= \Phi_{1}(X^7) 
\end{align*}

En analysant le processus récursif, on constate qu'une seule procédure récursive est suffisante pour effectuer ces calculs si elle possède un paramètre permettant de choisir une récursion sur un polynôme cyclotomique ou sur un polynôme cyclotomique inverse. \\

\textbf{Au terme de ce processus récursif, les seules opérations arithmétiques utilisées sont des additions et des soustractions.} En effet, dans les formules \ref{recurphi} et \ref{recurpsi}, on n'effectue effectivement que des multiplications par des polynômes de la forme $(1-X^d)$ et des divisions par de tels polynômes. Mais diviser par $(1-X^d)$, revient à multiplier par la série formelle $\displaystyle \sum_{k = 0}^\infty X^{kd}$ en s'arrêtant à un nombre de termes suffisants. C'est à dire $\varphi(n)$ pour le calcul de $\Phi_n(X)$ ou $n - \varphi(n)$ pour celui de $\Psi_n(X)$.\\



\hl{Analyser en détail le nombre de termes utilisés dans \texttt{SPS4} pour les produits par $(1-X^d)^{-1}$}.\\

\subsection*{Utilisation des packages listings et algorithmes}
\begin{lstlisting}
	# Test Latex python
	
	nterms = int(input("How many terms? "))
	
	# first two terms
	n1, n2 = 0, 1
	count = 0
\end{lstlisting}

\begin{algorithm}
	\caption{Procedure $\texttt{SPS4}(m, e, \lambda, D, D_{max}, a)$: multiplication par $\Phi_m(z^e)$ ou $\Psi_m(z^e)$ }\label{alg:cap} 
	\textbf{Input:}
	\begin{itemize}
		\item $m$: un entier impair sans carré, ordre du polynôme par lequel on multiplie
		\item $\lambda$: \texttt{TRUE} si l'on multiplie par un polynôme cyclotomique, \texttt{FALSE} par un polynôme cyclotomique inverse
		\item $a$: le polynôme $f$ courant que l'on va multiplier 
	\end{itemize}
	\begin{algorithmic}
		\Require $n \geq 0$
		\Ensure $y = x^n$
		\State $y \gets 1$
		\State $X \gets x$
		\State $N \gets n$
		\While{$N \neq 0$}
		\If{$N$ is even}
		\State $X \gets X \times X$
		\State $N \gets \frac{N}{2}$  \Comment{This is a comment}
		\ElsIf{$N$ is odd}
		\State $y \gets y \times X$
		\State $N \gets N - 1$
		\EndIf
		\EndWhile
	\end{algorithmic}
\end{algorithm}

\section*{Hauteur des polynômes cyclotomiques}\label{sec:hauteur}
Dans cette section, on étudie la magnitude des coefficients des polynômes cyclotomiques $\Phi_n(X)$. Dans un premier temps, pour $n$ premier ou produit de deux entiers premiers. Dans un second temps, on énoncera des résultats concernant la croissance asymptotique de la hauteur $A(n)$ des polynômes cyclotomiques.\\


On constate que pour $p$ premier:
$$\Phi_p(X) = \frac{X^p-1}{\Phi_1(X)} = \frac{X^p-1}{X-1} = \sum_{k=0}^{p-1} X^k,$$
et tous les coefficients de $\Phi_p(X)$ sont égaux à $1$. Si $n$ est produit de deux entiers premiers $p,q$, on va montrer que les coefficients du polynôme cyclotomique:
$$\Phi_{pq}(X) = \frac{(1-X)(1-X^{pq})}{(1-X^p)(1-X^q)}$$ sont dans $\{-1, 0, 1\}$. Le degré de $\Phi_{pq}(X)$ est égal à $(p-1)(q-1)$. En considérant l'égalité précédente dans l'anneau $\mathbb Z[[X]]$, où la division par $(1-X^p)(1-X^q)$ a un sens puisque le coefficient constant est égal à un, les coefficients de $\Phi_{pq}(X)$ sont donc ceux de $(1-X)(1-X^p)^{-1}(1-X^q)^{-1}$ tronqués à l'ordre $\varphi(pq)$. Prouvons tout d'abord que les coefficients $a_m$ de la série formelle
$$S_{pq}(X) = (1-X^p)^{-1}(1-X^q)^{-1} = \sum_{m \ge 0} a_m X^m$$ sont égaux à zéro ou un pour $0 \le m < pq$. Il suffit pour cela de montrer qu'il existe au plus un couple d'entiers $(a,b) \in \mathbb N^2$ tel que $ap+bq = m$ pour $0 \le m < pq$. Or le lemme de Gauss implique que si $ap+bq = a^\prime p + b^\prime q$, il existe $r$ entier tel que $a^\prime = a -rq$ et $b^\prime = b +rp$. Sans perte de généralité, on peut supposer $b^\prime \ge b$, c'est à dire $r \ge 0$. L'hypothèse $m <pq$ implique
$$pq > a^\prime p + b^\prime q \ge (b + r p)q \ge rpq$$ d'où $r=0$ et $(a,b)=(a^\prime, b^\prime)$.

Les coefficients de $\Phi_{pq}(X)$ qui sont donc ceux de $(1-X) S_{pq}(X)$ tronqués à l'ordre $\varphi(pq)$, sont différence de deux entiers de $\{0,1\}$ et appartiennent à $\{-1,0,1\}$, ce que nous souhaitions prouver. Cette démonstration s'appuie sur un sujet de préparation à l'agrégation \cite{agreg}.  Notons que la réciproque est fausse: $\Phi_{651} = \Phi_{3 \cdot 5 \cdot 31}$ est un polynôme cyclotomique d'ordre le produit de trois premiers distincts dont les coefficients sont dans $\{-1,0,1\}$.\\

On dit de $\Phi_{pq}$ dont l'ordre est produit de deux premiers distincts qu'il est binaire. Et de $\Phi_{pqr}$ qu'il est ternaire. $\Phi_{105} = \Phi_{3 \cdot 5 \cdot 7}$ est en quelque sorte le premier polynôme cyclotomique ternaire d'ordre impair. $-2$ est l'un de ses coefficients. N'importe quel entier est-il le coefficient d'un polynôme cyclotomique? La réponse est positive et on reproduit ici la preuve de cet article de Jiro Suzuki \cite{range}.\\

On note $C$ les valeurs prises par l'ensemble des coefficients de tous les polynômes cyclotomiques et on commence par prouver que
\begin{propriete}
	pour tout entier $t > 2$, il existe $t$ nombres premiers distincts $p_1 < p_2 < \cdots < p_t$ tels que $p_1+p_2 > p_t$.
\end{propriete}
Dans le cas contraire, il existerait un entier $t > 2$ tel que pour tous premiers vérifiant $p_1 < p_2 < \cdots < p_t$ on ait $p_1+p_2 \le p_t$ et donc $2 p_1 < p_t$. Cela implique que pour tout entier $k$, il y a au plus $t$ nombres premiers entre $2^{k-1}$ et $2^k$ et donc que $\pi(2^k) < kt$, contrairement à ce qu'affirme le théorème des nombres premiers à savoir que 
$$\pi(n) \underset{n \to \infty}{\sim} \frac{n}{\ln n}.$$

Revenons à notre affirmation initiale à savoir que tout entier $s \in \mathbb Z$ est le coefficient $c_i^{(n)}$ d'un polynôme cyclotomique $\Phi_n$. Soit $t$ un entier impair supérieur à 2 et $p_1 < p_2 < \cdots < p_t$ $t$ nombres premiers tels que $p_1+p_2 > p_t$. Notons $p = p_t$ et $n = p_1 p_2 \cdots p_t$. On a
$$\Phi_n(X) = \prod_{d \mid n} (1-X^d)^{\mu(\frac{n}{d})} = \sum_{i = 0}^{\varphi(n)} c_i^{(n)} X^i.$$
Pour $r > s$, on a $p_r + p_s > p_t = p$ et par conséquent
\begin{align*}
	\Phi_n(X) &\equiv \prod_{i=1}^{t} (1-X^{p_i})/(1-X) &(\mathrm{mod}X^{p+1})\\
	&\equiv (1 + X + \cdots + X^p)(1 - X^{p_1}+X^{p_2}+\cdots+X^{p_t})  &(\mathrm{mod}X^{p+1}).
\end{align*}
On constate que $c_p^{(n)}=-t+1$ et $c_{p-2}^{(n)}=-t+2$ ce qui montre que $\{s \in \mathbb Z \mid s \le -1\} \subseteq C$ puisque $t$ est n'importe quel nombre naturel impair supérieur ou égal à $3$.\\

On sait d'autre part que pour $m$ entier positif impair
$$\Phi_{2m}(X) = \Phi_m(-X).$$
Si $p_1 \ge 3$, $n = p_1 p_2 \cdots p_t$ est impair, $c_p^{(2n)}=t-1$ et $c_{p-2}^{(2n)}=t-2$. On en déduit que $\{s \in \mathbb Z \mid s \ge 1\} \subseteq C$. Et finalement que $C = \mathbb Z$ puisque $c_2^{15} = 0$.\\

Dans la preuve ci-dessus, on augmente indéfiniment le nombre de facteurs premiers $t$ de $n$ pour exhiber un coefficient arbitrairement grand d'un polynôme cyclotomique $\Phi_n$. Emma Lehmer démontra en 1936 \cite{infini} que les coefficients des polynômes cyclotomiques ternaires forment un ensemble infini. Pour montrer que $\limsup\limits_{n \to \infty} A(n) = \infty$, il "suffit" de considérer les polynômes cyclotomiques ternaires.\\

Le mathématicien suédois Carl Severin Wigert a démontré que
$$\limsup\limits_{n \to \infty} \frac{\log d(n)}{\log n / \log \log n} = \log 2$$ où $d(n)$ désigne le nombre de diviseurs de $n$; voir par exemple le théorème 317 de  \cite{numbers}. On utilise ce résultat pour prouver (en suivant le mathématicien américain Paul T. Bateman) que

\begin{propriete}
	$$\limsup\limits_{n \to \infty} A(n) \le \exp(n^{\log 2 / \log \log n}).$$
\end{propriete}

Partons de nouveau de l'égalité $\Phi_n(X) = \prod_{d \mid n} (1-X^d)^{\mu(\frac{n}{d})}$ et remarquons que
\begin{itemize}
	\item la hauteur d'un polynôme $P$ produit de polynômes $P_i = \sum_{j=0}^{n_i} p_j^{(i)} X^j$ est inférieure ou égale à la hauteur du polynôme $Q$ produit des polynômes $\widetilde{P_i} = \sum_{j=0}^{n_i} \lvert p_j^{(i)} \rvert X^j$,
	\item $1/(1-X^d)$ est égale à la série formelle $\sum_{k \ge 0} X^{kd}$, et les termes de degré supérieurs à $\varphi(n)$, donc supérieurs ou égaux à $n$ ne contribuent pas au calcul de $\Phi_n$.\\
\end{itemize}
On déduit de cela que la hauteur de $\Phi_n$ est inférieure ou égal à celle du polynôme
$$\Delta_n(X) = \prod_{d \mid n} (1 + X^d + X^{2d} + \cdots + X^{(n/d - 1)d}).$$
La hauteur d'un produit de polynômes dont les coefficients sont tous égaux à un étant moindre que le produit des valeurs de ces polynômes en un (ce que l'on prouve par récurrence sur le nombre de polynômes), on obtient en utilisant le résultat de Carl Severin Wigert mentionné plus haut

\begin{align*}
	A(n) &< \prod_{d \mid n} (n/d) = n^{d(n)/2} = \exp\left(\frac{1}{2}d(n) \log n\right)\\
	&< \exp\left(\frac{1}{2}2^{(1+\epsilon/2)\log n/ \log \log n}\log n\right)\\
	&< \exp\left(2^{(1+\epsilon)\log n/ \log \log n}\right)\\
	&= \exp\left(n^{(1+\epsilon)\log 2/ \log \log n}\right)
\end{align*}

pour tout $\epsilon > 0$ et $n$ suffisamment grand, ce qui permet de conclure.\\

Le mathématicien britannique Robert Charles Vaughan a prouvé que pour une infinité d'entiers $n$
$$A(n) > \exp\left(n^{\log 2/ \log \log n}\right).$$




\subsection*{Éléments clefs}
\begin{itemize}
	\item Utilisation de la formule d'inversion de Mobiüs.
	\item Et des polynômes cyclotomiques inverses.
	\item Formules (LEMME 1) liant les polynômes cyclotomiques et polynômes cyclotomiques inverses pour les produits d'un entier par une entier premier. Ces formules font appel à la division de polynômes.
	\item Pour la division de polynômes, il est possible d'utiliser un algorithme FFT. L'article décrit des calculs dans des corps de cardinal premier et reconstruction par le théorème des restes chinois. Il mentionne \textit{"For even though the numerator is sparse, the denominator and quotient are typically dense."}
	\item Analyser les problèmes provenant de la taille des entiers manipulés. En particulier, \texttt{Python} utilise pour les entiers une arithmétique sans limite de taille. \textit{Quelle est l'implication pour les performances?}
	\item La formule 3.14 est importante, car elle permet de transformer une division par un polynôme cyclotomique en un produit par un polynôme cyclotomique inverse et une division par un polynôme de la forme $1 - z^m$.
	\item Et en utilisant des résultats de séries formelles, cela revient à effectuer une multiplication par une série $1 + z^m + z^{2m} + \dots$... où il est nécessaire de contrôler le nombre de termes à conserver de la série.
	\item Les propriétés palindromiques des polynômes cyclotomiques permettent quant à elle de diviser par deux le nombre de coefficients à calculer. Le LEMME 4 étend ces propriétés aux produits de polynômes cyclotomiques (et de polynômes cyclotomiques inverses).
	\item Ces propriétés peuvent-être utilisées dans tous les calculs intermédiaires.
	\item Les formules 3.17 et 3.19 indiquent comment calculer les polynômes cyclotomiques (et cyclotomiques inverses) à partir des précédents.
	\item Pour de "très grands" polynômes cyclotomiques, on peut être amené à calculer "trop de termes" $\implies$ voir remarque 3.22.
	\item \textit{Analyser le dernier algorithme récursif!!!}
\end{itemize}

\subsection*{Algorithmes de l'article}

\subsection*{Implémentation SAGE}
SAGE implémente la méthode  \texttt{cyclotomic\_coeffs} \href{https://github.com/sagemath/sage/blob/develop/src/sage/rings/polynomial/cyclotomic.pyx}{ici}. Lorsque la hauteur du polynôme à calculer dépasse:
\begin{verbatim}
	cdef long fits_long_limit = 169828113 if sizeof(long) >= 8 else 10163195
\end{verbatim}
SAGE bascule sur un algorithme (plus lent) basé sur \href{https://pari.math.u-bordeaux.fr/}{PARI/GP} en précision infinie.

\subsection*{Implémentation SymPy}
De son côté, SymPy implémente la méthode \href{https://docs.sympy.org/latest/modules/polys/internals.html#sympy.polys.factortools.dup_zz_cyclotomic_poly}{dup\_zz\_cyclotomic\_poly}, à analyser. Il y a visiblement une arithmétique en précision infinie, basée sur gmpy2, \textit{comment s'effectue le basculement?}

\begin{thebibliography}{9}
	\bibitem{agreg}
	Pierre Charollois (2015) \href{https://webusers.imj-prg.fr/~pierre.charollois/Charollois_Pbme_cyclotomiques_agreg2015.pdf}{\emph{Problème de Mathématiques Générales no. 2.}}, Université Pierre et Marie Curie.
	
	\bibitem{range}
	Jiro Suzuki (1987) \href{https://projecteuclid.org/journals/proceedings-of-the-japan-academy-series-a-mathematical-sciences/volume-63/issue-7/On-coefficients-of-cyclotomic-polynomials/10.3792/pjaa.63.279.full}{\emph{On Coefficients of Cyclotomic Polynomials}}, Department of Mathematics, Sophia University
	Wesley, Massachusetts, 2nd ed.
	
	\bibitem{infini}
	Emma Lehmer (1936) \href{https://projecteuclid.org/journals/bulletin-of-the-american-mathematical-society-new-series/volume-42/issue-6/On-the-magnitude-of-the-coefficients-of-the-cyclotomic-polynomial/bams/1183498920.full}{\emph{On the magnitude of the coefficient of the cyclotomic polynomial}}, Bull. Amer. Math. Soc. 42(6): 389-392.
	
	\bibitem{numbers}
	G. H. Hardy et E. M. Wright (1936)
	\emph{An introduction to the theory of numbers}, Oxford University Press 4th edition.
\end{thebibliography}

\end{document}