
\documentclass{beamer}
\usetheme{Warsaw}
\usepackage[T1]{fontenc}
\usepackage[utf8]{inputenc}
\usepackage[french]{babel}
\usepackage{amsmath,latexsym}
\usepackage{amssymb}
\usepackage{bbm}
\usepackage{tikz}

\setbeamercovered{transparent}

\newcommand{\R}{\mathbbm{R}}
\newcommand{\Pv}{\mathbbm{P}}
\newcommand{\Ps}{\mathcal{P}}
\newcommand{\Rs}{\mathcal{R}}
\newcommand{\N}{\mathbbm{N}}
\newcommand{\Fs}{\mathbbm{F}}
\newcommand{\G}{\mathcal{G}}
\newcommand{\I}{\mathcal{I}}
\newcommand{\T}{\mathcal{T}}
\newcommand{\Ns}{\mathcal{N}}
\newcommand{\Z}{\mathbbm{Z}}
\newcommand{\Zs}{\mathcal{Z}}
\newcommand{\Q}{\mathbbm{Q}}
\newcommand{\Qs}{\mathcal{Q}}
\newcommand{\Fpol}{\mathcal{F}_{Pol}}
\newcommand{\Flin}{\mathcal{F}/_{\equiv}}
\newcommand{\F}{\mathcal{F}}
\newcommand{\Ls}{\mathcal{L}}
\newcommand{\A}{\mathrm{A}}
\newcommand{\Ss}{\mathcal{S}}
\newcommand{\B}{\mathcal{B}}
\newcommand{\E}{\mathcal{E}}
\newcommand{\K}{\mathcal{K}}
\newcommand{\C}{\mathcal{C}}
\newcommand{\U}{\mathcal{U}}
\newcommand{\M}{\mathcal{M}}
\newcommand{\Bo}{[}
\newcommand{\Bf}{]}
\newcommand{\ind}{\textrm{ind}}
\newcommand{\dom}{\mbox{dom }}
\newcommand{\ZFC}{\textbf{\mbox{ZFC}}}
\newcommand{\CD}{\textbf{\mbox{CD}}}
\newcommand{\PfN}{\mathcal{P}_f(\N)}
\newcommand{\PcofE}{\mathcal{P}_{cof}(E)}
\newcommand{\OO}{\mathcal{O}}
\newcommand{\inp}{\in_{\varphi}}
\newcommand{\pgcd}{\mathrm{pgcd}}


\title{TER - Calcul rapide des polynômes cyclotomiques}
\author{Jean-Philippe Merx}
\institute{M1 Mathématiques - Sorbonne Université}
\date{Mai 2025}


\begin{document}
	
	\frame{\titlepage}
	
	\mode<handout>{
		\begin{frame}
			\frametitle{Table des matières}
			\tableofcontents
		\end{frame}
	}
	
	\newtheorem{theoreme}{Théorème}
	\newtheorem{proposition}{Proposition}
	
	\section{Introduction}
	
	\begin{frame}{Contenu du TER}
		\begin{itemize}
			\item Analyser l'article d'Andrew Arnold et Michael Monagan sur le calcul rapide des polynômes cyclotomiques
			\item Expliquer les méthodes et algorithmes utilisés
			\item Éventuellement réaliser une mise en œuvre et utiliser des logiciels de calcul formel
		\end{itemize}
	\end{frame}
	
	\begin{frame}{Définition des polynômes cyclotomiques}
		\begin{itemize}
			\item Si $U_n$ est le groupe cyclique des racines $n$-ième complexes de l'unité, le polynôme cyclotomique $\Phi_n$ est:
			$$\Phi_n(X) = \prod_{\zeta \in U^*_n} (X - \zeta) = \prod_{\substack{j=1\\ \pgcd(j,n)=1}}^n (X - e^{\frac{2 \pi i}{n}j})$$ où $U_n^* \subseteq U_n$ est l'ensemble des générateurs de $U_n$.
			\item Polynôme cyclotomique inverse:
			$$\Psi_n(x) = \prod_{\zeta \in U_n \setminus U^*_n} (X - \zeta) = \prod_{\substack{j=1\\ \pgcd(j,n) > 1}}^n (X - e^{\frac{2 \pi i}{n}j}) = \frac{X^n - 1}{\Phi_n(X)}.$$
		\end{itemize}
	\end{frame}
	
	\begin{frame}{De l'importance des diviseurs de $n$}
		\begin{itemize}
			\item $\Phi_n \in \mathbb Z[X]$ de degré $\varphi(n)$ où $\varphi$ est l'indicatrice d'Euler et:
			$$P_n(X) = X^n-1 = \prod_{ d \mid n} \Phi_d(X)$$
			\item Pour $p$ premier ne divisant pas $n$:
			$$\Phi_{np}(X) = \frac{\Phi_n(X^p)}{\Phi_n(X)} \text{ et } \Psi_{np}(X) = \Psi_n(X^p)\Phi_n(X)$$
			et pour $q$ premier divisant $n$
			$$\Phi_{nq}(X) = \Phi_n(X^q) \text{ et } \Psi_{nq}(X) = \Psi_n(X^q)$$
		\end{itemize}
	\end{frame}

	\section{Un premier algorithme}

	\begin{frame}{Restriction à $n$ produit de premiers impairs distincts}
		\begin{itemize}
			\item Comme on a aussi: $$\Phi_{2n}(X) = \Phi_n(X^2) \text{ si } 2 \mid n \text{ et }\Phi_{2n}(X) = \Phi_n(-X) \text{ sinon}$$ $\implies$ il suffit de considérer $n$ produit de premiers impairs distincts.
			\item D'où un premier algorithme possible pour $n = p_1^{e_1} \cdots p_k^{e_k}$:
			\begin{itemize}
				\item Itérations pour calculer $\Phi_m(X) = \Phi_{p_1 \cdots p_k}(X)$
				\item Calcul de $\Phi_{n/m}(X)$
			\end{itemize}
			\item Les divisions de polynômes sont la base de l'algorithme
			\item Utilisation d'une division rapide avec Newton sur séries formelles et FFT
		\end{itemize}
	\end{frame}

	\section{Formules closes et algorithmes SPS}

	\begin{frame}{Formules closes}
		Si $\mu : \mathbb N^* \to \{-1, 0, 1\}\}$ est la fonction de Möbius:
		\begin{displaymath}
			\mu(n) = :
			\begin{cases}
				1 & \text{si } n = 1\\
				0 & \text{si } n \text{ a un facteur premier carré}\\
				(-1)^r & \text{où } $r$ \text{ est le nombre de facteurs premiers de } $n$\\
			\end{cases}
		\end{displaymath}
		on a:
		\begin{align}		
			\Phi_n(X) = \prod_{d \mid n} (1 - X^\frac{n}{d})^{\mu(d)} \tag{2.4}\label{mobphi}\\
			\Psi_n(X) = -\prod_{d \mid n, d<n} (1 - X^d)^{-\mu(\frac{n}{d})}\tag{2.5}\label{mobpsi}
		\end{align}
	\end{frame}

	\begin{frame}{Points critiques pour l'implémentation}
		\begin{itemize}
			\item \textbf{RAPIDITÉ} $\implies$ type d'algorithme
			\item \textbf{Taille mémoire}: pour $n = 3 \cdot 5 \cdot 7 \cdot 11 \cdot 13 \cdot 17 \cdot 19 \cdot 23 \cdot 29$, $\deg \Phi_n = 1,021,870,080$ et on obtient un polynôme nécessitant au moins 8 Go de mémoire.
			\item \textbf{Hauteur} $A(n)$: quel type de données manipuler?\\
			Un sujet en soi sur lequel on revient plus loin		
		\end{itemize}
	\end{frame}

	\begin{frame}{Algorithmes SPS et SPS-Psi}
		\begin{itemize}
			\item 
			Application des formules closes:
			\begin{itemize}
				\item Multiplication par $1-X^d$: $\varphi(n)$ soustractions
				\item Division par $1-X^d$... ATTENTION!!!\\
				Seules $\varphi(n)$ additions sont nécessaires et non $\left(\varphi(n)\right)^2$
			\end{itemize}
			\item Algorithme simple avec une allocation mémoire unique pour $\Phi_n$
			\item... mais tous les calculs sont effectués sur un tableau de taille $\varphi(n)$
		\end{itemize}		
		
	\end{frame}

	\begin{frame}{Algorithme SPS4}
	Utilisation de la localité des formules initiales et de l'efficacité des formules closes\\
	
	Formules récursives de calcul des polynômes cyclotomiques:	
	\begin{align*}
		\Phi_n(X) &=\prod_{j=2}^{k} - \Psi_{m_j}(X^{e_j}) \prod_{j=1}^{k} (1-X^{n/p_j})^{-1}(1-X^n)\tag{3.17}\label{recurphi}\\
		\Psi_{n}(X) &=\prod_{j=1}^{k} \Phi_{m_j}(X^{e_j})\tag{3.25}\label{recurpsi}
	\end{align*}
	où $n = p_1 \cdots p_k$ et pour $1 \le j \le k$, $e_j = p_{j+1} \cdots p_k$ et $m_j = p_1 \cdots p_{j-1}$\\
	\textbf{Calcul sur des polynômes de degrés inférieurs au degré total}
	
	\end{frame}

	\begin{frame}{Implémentations SPS}
		Implémentations étudiées:
		\begin{itemize}
			\item 
			\textbf{SAGE} implémente directement les formules closes en Python + C pour le cœur des calculs
			\item \textbf{SYMPY}(très lent) en reste aux divisions
			\item Arnold et Monagan implémentent SPS4 en C (\textit{je n'ai pas testé leur programme})
			\item \textbf{"La mienne"} implémente SPS4 en Python avec allocation mémoire pour chaque polynôme
		\end{itemize}
	\end{frame}

	\section{Hauteur $A(n)$ des polynômes cyclotomiques}
	
	\begin{frame}{Importance de la hauteur dans un algorithme}
		Choix à effectuer:
		\begin{itemize}
			\item Entiers de taille fixe ou pas?
			\item Pré-calcul ou pas par un algorithme plus lent des $n$ qui obligent à changer de format d'entiers?
			\item \textbf{Et quand est-il des calculs intermédiaires?}
			\item Test de l'overflow au cours des calculs
		\end{itemize}
		Options possibles:
		\begin{itemize}
			\item Limiter $n$...
			\item Changer de représentation en cours de calculs
			\item Calculs modulo des premiers et reconstitution
			\item Entiers de longueur infinie
		\end{itemize}
	\end{frame}

	\begin{frame}{Résultats théoriques sur $A(n)$}
		\begin{itemize}
			\item Pour $p < q$ premiers $A(pq) = 1$
			\item $\limsup\limits_{n \to \infty} A(n) = \infty$ et les polynômes cyclotomiques ternaires suffisent
			\item Tout entier est le coefficient d'un polynôme cyclotomique
			\item $\limsup\limits_{n \to \infty} A(n) \le \exp(n^{\log 2 / \log \log n})$
			\item Pour une infinité d'entiers $n$, $A(n) > \exp(n^{\log 2 / \log \log n})$
		\end{itemize}		
	\end{frame}
		
	
	\section{Démonstration}
	\begin{frame}{Démonstration}
		
		\href{https://jupyter.math.sorbonne-universite.fr/user/21304439/lab}{Notebooks Python SAGE \& module Python}
		
	\end{frame}
	
\end{document}
